\chapter{基于卷积神经网络的脑电P300检测方法}

对于非侵入式脑机接口, 已有很多基于传统机器学习方法的相关工作在P300检测方面进行研究。 深度学习方法凭借可训特征(或可训的kernel)优势,在语音识别, 图像处理, 声音合成, 生物信号分析等方面超过了手工定义特征的传统方法, 借助现代计算机计算能力的提高和大数据量的爆发得以广泛应用。 本章中, 也采用深度学习的方法进行EEG信号建模, 提高其对P300波形检测效果。 已有相关工作利用卷积神经网络进行建模并达到较好的P300检测效果, 但在其模型中有很多部分可以进行改进。 在我们的ConvP300Net方法中, 主要通过改进模型结构, 更好地利用信号空间相关性; 通过重新设计对数据不均衡问题的处理方法, 可以使模型更好地拟合所用数据, 而不需考虑模型集成问题带来的识别增益等。 最后在实验中, ConvP300Net与之前基于卷积神经网络进行P300检测的相关工作进行比较, 效果提升7\%。


\section{深度学习介绍}

\subsection{传统机器学习方法}

在之前的50年左右,传统的模式识别模型用手工定义的特征进行特征提取,通过对数据的分析选取可训分类器进行模型构建。 最近10年,借助现代计算机计算能力的提高和大数据量的爆发,神经网络方法得以重新广泛应用,在很多领域都达到非常好的效果, 我们称这种利用大规模网络进行模式识别方法为深度学习(Deep Learning), 也叫End-To-End Learning。不同于传统模型采用固定特征,或者固定kernel(核函数)进行样本度量; 深度学习采用可训特征(或可训的kernel), 然后将特征作为可训练的分类器输入, 进行训练, 如表\ref{Tab:dl_overview_compare}。

\begin{table}[ht]
\centering
  \begin{tabular}{|c||c|c|c|}
  \hline
   & 特征 & 分类器 & 特点\\
  \hline\hline
   传统方法 & 人工定义的特征 & 简单可训练分类器 & 特征设计费时,需强业务背景\\
  \hline
  深度学习 & 训练特征提取模型 & 复杂可训练分类器 & End-to-End learning,feature易操作\\
  \hline
  \end{tabular}
  \caption{深度学习与传统模式识别方法}
  \centering \label{Tab:dl_overview_compare}
\end{table}

历史上,第一个有学习功能的机器为1960年提出的Perceptron\cite{rosenblatt1960perceptron},也是神经网络的一个基本单元。 Perceptron是一个简单特征提取器上加载的一个线性分类器: 


\begin{equation}
\label{Eq:Perceptron}
y=sign(\sum_{i=1}^N{w_iF_i(x)}+b)
\end{equation}


其中$x$为数据,$F_i(x)$为x的第i个特征,$w_i$为相应的特征权重参数, b为常参数, $sign$为分类器的非线性函数,对于二类分类,$sign$函数将结果映射到(0,1)。

\begin{figure*}[htb]
  \centering
  % Requires \usepackage{graphicx}
  \includegraphics[scale=0.9]{Pictures/CNN/perceptron.jpg}\\
  \caption{Perceptront图例}\label{fig:perceptron}
\end{figure*}

目前最普及的实际应用也用到了线性分类器的一些变种,或者叫模板匹配(template matching)。 但是由于其底层的特征提取器需要反映特定信号的特点, 所以需要由特定领域专家来设置。 比如图像处理领域, 对于不同任务( 图像分类, 图像分割, 图像跟踪等), 所要求的特征就各不相同, 需要针对特定任务定义图像特征。 此外,传统方法也很难设计kernel, 从而不容易表达对距离的度量,仍然以图像来说,最简单的距离度量思路是对应像素相减,但是这显然不能表达图像语义层的相似信息。

为了能使特征更加灵活而且不过多依赖于专家指定的特征, 很多方法提出, 可以先人为定义简单特征, 然后通过无监督学习方法进一步得到中间层特征, 将其输入分类器进行最后分类。
典型的无监督特征学习方法如混合高斯模型\cite{jeong2004image,gray2001gauss,gauvain1992bayesian,reynolds2000speaker,reynolds1995robust, zivkovic2004improved,lee2005effective,yang1999gaussian}, K-Means\cite{liu2007computational,netzer2011reading,coates2011text,dy2004feature,coates2012learning}, 和Sparse Coding\cite{yang2009linear,boureau2008sparse,coates2011importance,coates2011analysis,gao2010local,mairal2010online}。 但这仍然不能解决以下几个问题:

\begin{enumerate}
	\item 建立传统模型代价大\\
	对每个领域,每个任务都需要设计特征。即便有了中间特征层进行无监督的特征选择, 庞大冗余的基础特征的设计也是耗时费力的。 随着工业界对不同任务的需求, 需要建立很多基础特征、 模型, 代价也很大。
	\item 无法很好地利用计算性能\\
	计算机性能的大幅提升本可以用来帮助加速机器学习, 但传统机器学习需要人工定义模型, 从而使模型规模受限,不能很好地利用计算性能和大量数据。 
	
	\item 人工定义特征效果欠佳\\
	目前, 自动学习的特征已经在图像、语音等很多领域强于人工定义的特征。 而且如果需要增加特征维度进行大规模学习就需要再手工定义更多特征, 而不能简单地够自动按比例扩大。
	
\end{enumerate}


\subsection{深度学习方法}

深度学习是近几年来很热的机器学习算法, 在语音识别,计算机视觉,自然语言处理等领域打破了维持了多年的竞赛记录。以最典型的时序模型——语音识别的发展为例, 在1980年代早期, 主要应用Dynamic time Warping (DTW)\cite{juang1984hidden,myers1981comparative,rabiner1978considerations,berndt1994using}, 输入底层特征,通过无监督学习所得中间层特征输入分类器进行分类。 1985年后, \cite{huang1990hidden,rabiner1989tutorial,rabiner1986introduction}提出在中间层用隐马尔科夫模型描述一个序列出现的概率,得到进一步改进。 2010年左右, 使用深度神经网络进行逐层有监督学习达到最佳效果\cite{waibel1989modular,nakamura1989speaker}。


同传统方法的基本方法类似, 深度学习也是从数据分别生成底层特征, 中间层特征, 最后加入高层特征, 输入分类器进行分类。 其形式化表示为:
\begin{equation}
y=F(W^kF(W^{k-1}F(...F(W^0X)...)))
\label{Eq:dl_formulation}
\end{equation}
其中, W为权重参数,k为层数,$W^iX$为输入X的线性表示, $F$为非线性函数。 深度学习就是优化$y$与groundtruth之间的距离,使之最小化。 从图像角度, 如图\ref{fig:feature_level}所示\cite{zeiler2014visualizing,zeiler2013stochastic},图中(a),(b),(c)分别为自动学得的底层特征,中层特征和高层特征, 其中底层特征学习图像的浅层特征,在所有类别中共享, 如(a)中,学到的特征类似Gabor滤波器所提取的边缘特征\cite{ngiam2010tiled,shi1998gabor}, 从中间层到高层依次学习图像更深层的语义特征, 如有语义的显著图像区域, 高层特征更为稳定,也具有类属性。  从自然语言处理的角度, 初始输入为字符, 从底层向上依次学习单词, 短语, 长句, 文章。

\begin{figure*}[htb]
  \centering
  % Requires \usepackage{graphicx}
  \includegraphics[scale=0.9]{Pictures/CNN/low_mid_high_feature.png}\\
  \caption{各层次feature}\label{fig:feature_level}
\end{figure*}


除了自动学习特征外, 深度学习和传统模型还有一个重要区别就是“深”。 一般来说, 深度结构由多层含参的非线性模型组成, 以形成特征的层次结构。
在\cite{bengio2007scaling,bengio2009learning}中讨论了深度模型的必要性。 浅层结构以现在的kernel machine\cite{scholkopf1999advances}为例, 如Support Vector Machine(SVMs)\cite{boser1992training,cortes1995support}。 这些方法定义特征
为一系列kernel函数连接而成的向量, 在训练数据中进行模板匹配, 如式\ref{Eq:kernel_fea_vec}所示。

\begin{equation}
	\label{Eq:kernel_fea_vec}
	\varphi(x) = [k(x,\mu_1),k(x,\mu_2),...,k(x,\mu_n)]
\end{equation}

式中, $\mu_i$为数据样本的一部分, 即模板样本; $\varphi(x)$为样本$x$的特征。 而后$\varphi(x)$通过线性组合进行分类:


\begin{equation}
	\label{Eq:kernel_predict}
	F(x) = W^T\varphi(x)
\end{equation}

可见, kernel方法就是一个简单的模板匹配层连接一个线性函数层, 由于方法中模板都是从原始训练数据中提取而得, 所以kernel machine的第一层
可视作一种简单的无监督方法, 只有式\ref{Eq:kernel_predict}中参数$W$的学习为有监督部分,没有涉及特征的层次结构, 因此不是深度模型。 同样,只有一个隐层的模型(如multilayer perceptron)也不算深度模型。 又如分类回归树(Classification and Regression Tree, CART), 其中所有决策都是在输入空间定义的, 同样没有根据特征的层次结构进行学习, 因此也不属于深度模型。 虽然有一些理论保证浅层结构可以很好地拟合任何复杂函数, 它们却无法保证特征的高效表征。 而深度模型可以从根本上通过简单设置网络结构更有效地表示特定函数。 最典型的深度结构就是有多个隐层的神经网络, 其中每个隐层节点之间的连接权重(如图\ref{fig:perceptron}中的$w_i$)都是通过有监督学习而来的, 因此特征具有层次结构。


由于90年代计算资源和数据量受限, 当隐层层数大于2的时候深度网络训练困难\cite{tesauro1992practical}, 所以之前神经网络的研究工作主要集中于浅层结构。 随着现在数据量的爆发, 神经网络的研究变得重新热门起来。 从网络结构来分, 有前向网络, 回馈网络和双向网络三类\cite{lecun2013deep}。 其中卷积神经网络属于前向网络, 广泛应用于图像识别。 我们将在第\ref{sec:CNN}节中介绍卷积神经网络的结构与训练过程, 在第\ref{sec:cnn_configure}节中介绍我们的数据与网络配置, 最后在第\ref{sec:cnn_experiment}中给出实验结果。


\section{卷积神经网络}\label{sec:CNN}

\subsection{Perceptron}

\subsection{CNN网络结构}


\section{ConvP300Net}\label{sec:cnn_configure}

\subsection{任务描述}

\subsection{网络结构}

\section{实验结果}\label{sec:cnn_experiment}















%\chapter{卷积神经网络}



%
%\input{./Chapters/chap3_2}
%
%\section{插入算法伪代码}
%以下是插入算法伪代码示例,如算法\ref{alg:skeleton_dt}所示。
%
%\begin{algorithm}[!htb]
%\caption{基于距离变换的骨架提取}
%\label{alg:skeleton_dt}
%\begin{algorithmic}[1]
%\Require 前景的二值图 bw \Comment{像素的灰度值为0或1} 
%\Ensure 骨架图 skel
%\State // 第1次遍历:从上往下,从左往右
%\For{$i=1,\dots ,M$} \Comment{M是二值图的高度}
%    \For{$j=1,\dots ,N$} \Comment{N是二值图的宽度}
%        \State bw[i][j] = 1 + min(bw[i][j-1], bw[i-1][j])  \Comment{min函数取极小值} 
%    \EndFor
%\EndFor
%\State // 第2次遍历:从下往上,从右往左
%\For{$i=M,\dots ,1$}
%    \For{$j=N,\dots ,1$}
%        \State bw[i][j] = 1 + min(bw[i][j], bw[i+1][j], bw[i][j+1])
%    \EndFor
%\EndFor
%\State // 第3次遍历:获取骨架图
%\State skel 的空间分配,并将每个像素初始化为0
%\For{$i=1,\dots ,M$}
%    \For{$j=1,\dots ,N$}
%        \State t = max(bw[i-1][j], bw[i+1][j], bw[i][j-1], bw[i][j+1])  \Comment{max函数取极大值} 
%        \State t = max(t, bw[i-1][j-1],bw[i-1][j+1],bw[i+1][j-1],bw[i+1][j+1])
%        \If {bw[i][j]>=t}
%        	\State skel[i][j]=1 \Comment{骨架点} 
%        \Else 
%        	\State skel[i][j]=0
%        \EndIf
%    \EndFor
%\EndFor
%\end{algorithmic}
%\end{algorithm}
%
%\section{插入源代码}
%代码示例:
%
%\begin{lstlisting}[language=C] 
%int main(int argc, char ** argv) 
%{ 
%	printf("Hello world! \n"); 
%	return 0; 
%} 
%\end{lstlisting} 
%
%\section{插入定义、定理等}
%实例来自 https://github.com/eclipselu/zjuthesis-mphil。
%
%\begin{hypo}
%待月西厢下,迎风户半开;隔墙花影动,疑是玉人来。
%\begin{eqnarray}
%  \label{eq:eqnxmp}
%  c & = & a^2 - b^2\\
%    & = & (a+b)(a-b)
%\end{eqnarray}
%\end{hypo}
%
%\begin{defin}
%子曰:「道千乘之国,敬事而信,节用而爱人,使民以时。」
%\end{defin}
%
%\begin{theo}
%犯我强汉者,虽远必诛。\hfill —— 陈汤(汉)
%\end{theo}
%
%\begin{pro}
%天不言自高,水不言自流。
%\begin{gather*}
%\begin{split} 
%\varphi(x,z)
%&=z-\gamma_{10}x-\gamma_{mn}x^mz^n\\
%&=z-Mr^{-1}x-Mr^{-(m+n)}x^mz^n
%\end{split}\\[6pt]
%\begin{align} \zeta^0&=(\xi^0)^2,\\
%\zeta^1 &=\xi^0\xi^1,\\
%\zeta^2 &=(\xi^1)^2,
%\end{align}
%\end{gather*}
%\end{pro}
%
