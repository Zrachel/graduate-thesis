\chapter{总结}

本篇论文围绕脑机接口相关课题做出研究。 第二章中对植入式脑机接口在运动皮层采集到的信号进行压缩。 由于目前大多数脑机接口信号的压缩工作集中在非植入式脑机接口, 而非植入式脑机接口捕获信号具有低采样率, 信号集中在低频部分等特性, 不能将其压缩方法直接应用于植入式脑机接口的压缩。 所以, 我们针对非植入式脑机接口所采集到的信号提出了一套压缩框架。 该框架中, 我们首先将数据转换到频域, 根据频域数据大小分为高幅值部分和低幅值部分。 用双阶编码方法分别对低幅值和高幅值信号进行符号编码和量化后混合无损编码。 在符号编码方法中, 我们用1-bit符号代替低幅值部分的每个幅值。 高幅值部分, 我们首先进行量化, 然后结合哈弗曼编码和零长编码进行混合编码, 在这里我们提出了一个算法计算混合编码中的分界线。 最后, 我们在实验中比较了我们的高保真压缩方法, 传统数据压缩方法和音频压缩方法。 

第三章和第四章中, 我们分别用卷积神经网络和循环神经网络对EEG信号进行建模, 并检测P300波形。 第三章中, 我们用一个6层的卷积神经网络ConvP300Net进行建模, 经过局部连接层, 卷积层, 全连接层, 分别从时域和空间域对数据进行卷积, 然后将P300检测转换为一个二类分类问题训练模型。 第四章中, 我们用长短时间记忆方法对P300波形进行建模。 在LSTM网络中, 每个节点为一个记忆块(memory block), 记忆块中以记忆单元为基本单位, 每个时刻的输出与该节点下一时刻的输入相连, 这样可以更好地利用网络的时序性。 在我们的实验中, 采用BCI数据集中的P300波形数据, 对P300波形进行检测, 并将其与之前的神经网络检测P300方法进行比较。 

在本篇论文中, 我们通过数据分析脑机接口所得数据, 更好地了解了植入式脑机接口与非植入式脑机接口信号的特性, 并通过深度学习方法进行P300检测, 识别率超过之前的神经网络分类方法。 此外, 通过深度学习方法所得的隐层数据也可以作为原EEG信号的特征表示用在其他任务中。 在第三、四章中我们没有进行P300波形分类, 而只进行了P300波形检测, 这是因为各类的分类数据有限, 如果每一类的数据充足, 也可以在后续进行数据分类。 对于数据压缩部分, 由于实验设备有限, 我们也只在运动神经元上采集的植入式信号进行压缩, 但由于压缩时没有运用运动皮层信号的通道间相关性, 所以该压缩方法将来可以应用到其他高分辨率神经信号的压缩上, 另外, 如果可以挖掘其他信号其他的通道间相关性, 还可以进一步提升压缩效果。


