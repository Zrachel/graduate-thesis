\chapter{总结}

本篇论文围绕脑机接口相关课题做出研究。 分别研究了侵入式脑机信号压缩与基于EEG的脑电P300波形检测。 在侵入式脑机信号压缩中, 提出了双阶编码方法, 并实现了一套完整的压缩/解压框架; 在脑电P300波形检测方法探索中, 我们采用神经网络对EEG信号进行建模并建立其与P300信号之间联系。 主要结论如下:

\begin{enumerate}
\item 针对非侵入式脑机接口所采集到的信号提出了一套压缩框架。 该框架中, 我们首先将数据转换到频域, 并通过幅值滤波器, 用双阶编码方法分别对滤波后不同部分的信号进行符号编码和量化后混合无损编码。 在符号编码方法中, 我们用1-bit符号代替低幅值部分的每个幅值。 高幅值部分, 我们首先进行量化, 然后结合哈弗曼编码和零长编码进行混合编码, 在这里我们提出了一个算法计算混合编码中的分界线。 最后, 我们在实验中比较了我们的高保真压缩方法, 传统数据压缩方法和音频压缩方法, 双阶编码方法可以在相同压缩率下达到更高的信号保真度, 在18\%的压缩比下达到36db的信噪比, 并保留了92\%的动作电位信号。

\item 我们分别用卷积神经网络和循环神经网络对EEG信号进行建模, 并检测P300波形。 ConvP300Net中用一个6层的卷积神经网络ConvP300Net进行建模, 经过局部连接层, 卷积层, 全连接层, 分别从时域和空间域对数据进行卷积, 然后将P300检测转换为一个二类分类问题训练模型。 LSTMP300Net中用长短时间记忆方法对P300波形进行建模。 在LSTM网络中, 每个节点为一个记忆块(memory block), 记忆块中以记忆单元为基本单位, 每个时刻的输出与该节点下一时刻的输入相连, 这样可以更好地利用网络的时序性。 在我们的实验中, 采用BCI数据集中的P300波形数据, 对P300波形进行检测, 并将其与之前的神经网络检测P300方法进行比较。 结果ConvP300Net和LSTMP300Net在BCI竞赛数据集上分别达到80.09\%和84.47\%的识别率, 均超过已有工作对P300波形的检测效果。

\end{enumerate}

在本篇论文中, 我们通过数据分析脑机接口所得数据, 更好地了解了侵入式脑机接口与非侵入式脑机接口信号的特性, 并通过深度学习方法进行P300检测, 结果超过一些基于神经网络分类工作的已有方法。 此外, 通过深度学习方法所得的隐层数据也可以作为原EEG信号的特征表示用在其他任务中。 后期工作中, 如果每一类的数据充足, 也可以用EEG信号对P300波形进行分类。 对于数据压缩部分, 由于实验设备有限, 我们也只在运动神经元上采集的侵入式信号进行压缩, 但由于压缩时没有运用运动皮层信号的通道间相关性, 所以该压缩方法将来可以应用到其他高分辨率神经信号的压缩上, 另外, 如果可以挖掘其他信号其他的通道间相关性, 还可以进一步提升压缩效果。


