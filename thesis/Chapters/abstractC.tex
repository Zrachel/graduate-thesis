\begin{abstract}

多年来人们在生物学医学领域的研究发现, 生物脑电活动可以通过一个非肌肉通路向外设传递信息, 这个外设被称作脑机接口。 脑机接口可以代替受损的神经系统,通过大脑信号采集,信号处理,解码为计算机指令,反馈这四步,为残障人士提供自动化服务。本论文主要研究侵入式脑电信号处理,压缩,以及EEG信号中的P300波形检测。

侵入式脑机接口在神经手术中直接植入大脑灰质, 其采集的信号具有高分辨率, 高采样率的性质,因此压缩对于信号存储和传输都是必不可少的环节。 在压缩方面,过去的神经信号压缩工作主要关注于非侵入式脑电信号,这些方法都应用了相关信号的特性。 但这些信号同侵入式信号有很大差异,所以其压缩方法不能直接套用在侵入式脑电信号中。 另一方面, 低分辨率的非侵入式脑机接口采样方便, 如脑电图(EEG)在大脑头皮非侵入式地记录神经元内离子电流产生的电位波动\cite{nie2005ele}, 所采集的信号在神经解码中广泛应用。 P300波形是大脑在决策时产生的事件相关电位中的正向偏移, 表示年轻成年人对简单的感官可分辨目标响应峰值延时约300ms。 

本文的主要工作如下:
\begin{enumerate}
\item 通过侵入式电极研究大脑运动皮层信号性质, 利用信道内部神经信号特性建立了一个完整的高保真运动皮层信号压缩框架。 该框架称为双阶编码方法, 创新地采用幅值滤波器对侵入式脑电信号的频域数据进行分割, 提出符号编码和混合编码方法分别对分割后数据进行压缩。 混合编码方法由哈弗曼编码和零长编码组合, 我们根据零分量占比分布提出了界限下降算法来求取两种编码方法的分界点。 在实验中, 我们将双阶编码方法与传统压缩方法和典型时序信号压缩方法进行了比较, 双阶编码方法可以在相同压缩率下达到更高的信号保真度, 在18\%的压缩比下达到36db的信噪比, 并保留了92\%的动作电位信号。

\item 采用深度学习的方法进行P300波形信号处理与检测。 文中, 我们分别建立了ConvP300Net和LSTMP300Net对EEG信号建模。 ConvP300Net中提出用一个6层卷积神经网络, 用局部连接层在空间维度对数据进行非共享卷积, 用卷积层对信号在时间维度进行共享卷积 , 用全连接层增强网络表达能力, 最后采用带权损失函数度量网络分类能力。 LSTMP300Net中采用一个5层网络, 其中3个中间层分别为长短时间记忆(LSTM)层。 不同于属于前向网络的卷积神经网络, LSTM层内各节点相互连接, 可以充分结合EEG信号在时间和空间维度的特性提高检测准确性,完善模型。 我们对ConvP300Net和LSTMP300Net分别用backpropagation和backpropagation through time方法进行训练, 并与相关工作进行比较, 结果ConvP300Net和LSTMP300Net在BCI竞赛数据集上分别达到80.09\%和84.47\%的识别率, 均超过已有工作对P300波形的检测效果。

\end{enumerate}





\keywords{脑机接口,压缩,卷积神经网络,长短期记忆方法,循环神经网络}
\end{abstract}
