\begin{abstract}
脑机接口(Brain Machine Interface, BCI)是一个完成大脑神经元和外部机器互联,通信与控制的接口。脑机接口可以代替受损的神经系统,通过大脑信号采集,信号处理,解码为计算机指令,反馈这四步,为残障人士提供自动化服务。本文主要研究植入式脑电信号处理,压缩,以及xx脑电信号的解码。植入式脑电信号具有高采样率,多通道,高分辨率的性质,因此压缩对于信号存储和传输都是必不可少的环节;另一个脑机接口的核心工作是神经信号的解码,即将神经信号翻译成运动指令,只有正确地解码才能实现BCI执行符合用户意愿。
在压缩方面,过去的神经信号压缩工作主要关注于非植入式脑电信号,如EEG(脑电图),EMG(肌电图)等,这些方法都应用了相关信号的特性。但这些信号同植入式信号有很大差异,它们的有效信息都在低频部分,可以直接通过高频滤波器获得有效信号,而植入式脑机接口获得的信号中高频部分包括神经元锋电位(spike),它能有效地进行神经活动解码,因而不能丢弃。所以,EEG等信号的压缩方法不能直接套用在植入式脑机接口获得的信号中。本文通过植入式电极研究大脑运动皮层信号性质,利用信道内部神经信号特性建立了一个完整的高保真运动皮层信号压缩框架,达到在维持信噪比为36db下压缩比例为18\%,并以92\%的保真度保存神经元锋电位信号,保证了重建效果。在解码方面,……本文中,我们将深度学习的方法应用于xx信号,分别在空间和时间维度建立深度神经网络,结合神经信号的时序性提高解码准确性,完善模型。
本文工作实现了植入式脑机接口神经信号的高保真压缩,xx信号的解码研究,主要创新点在于(1)对植入式脑电信号进行光谱分析,针对压缩问题提出了频谱-幅值压缩框架,有效地完成了运动皮层信号压缩,(2)通过神经网络对神经信号进行解码,利用了信号的空间结构与时序关系。


\keywords{脑机接口,压缩,卷积神经网络,长短期记忆方法,循环神经网络}
\end{abstract}
