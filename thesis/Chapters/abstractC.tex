\begin{abstract}
脑机接口(Brain Machine Interface, BCI)是一个完成大脑神经元和外部机器互联,通信与控制的接口。脑机接口可以代替受损的神经系统,通过大脑信号采集,信号处理,解码为计算机指令,反馈这四步,为残障人士提供自动化服务。本论文主要研究植入式脑电信号处理,压缩,以及EEG信号中的P300波形检测。

植入式脑电信号具有高采样率,多通道,高分辨率的性质,因此压缩对于信号存储和传输都是必不可少的环节。 在压缩方面,过去的神经信号压缩工作主要关注于非植入式脑电信号,这些方法都应用了相关信号的特性。 但这些信号同植入式信号有很大差异,它们的有效信息都在低频部分,可以直接通过低通滤波器获得有效信号,而植入式脑机接口获得的信号中高频部分包括神经元锋电位(spike),它能有效地进行解码,因而不能丢弃。所以,EEG等信号的压缩方法不能直接套用在植入式脑机接口获得的信号中。本文通过植入式电极研究大脑运动皮层信号性质,利用信道内部神经信号特性建立了一个完整的高保真运动皮层信号压缩框架。

另一方面, 低分辨率的非植入式脑机接口采样方便, 在信号解码中广泛应用。 只有正确地解码才能实现BCI执行符合用户意愿, 所以神经解码是BCI的核心任务, 我们在本文进行解码的前一步工作, P300波形信号处理与检测。 本文中, 我们采用深度学习的方法应用于P300波形, 分别用卷积神经网络和循环神经网络对EEG信号建模,结合EEG信号在时间和空间维度的特性提高检测准确性,完善模型。


\keywords{脑机接口,压缩,卷积神经网络,长短期记忆方法,循环神经网络}
\end{abstract}
