\begin{englishabstract}
Brain Machine Interface (BMI) is an interface that connect neurons of brain and devices to communicate and control. BMI can substitute impaired neural system, provide automatic service to disabled people through four steps: sampling signal from brain, signal processing, signal decoding to computer instruction, and feedback. This thesis focus on invasive electroneurographic signal processing, compression and P300 wavelet detection in Electroencephalograph signal. 

Invasive electroneurographic signal with high sampling rate and multiple channel can represent high-resolution information. However, the huge quantity of data makes compression indispensable. 
In the aspect of signal compression, some related work focus on non-invasive electroneurographic signal. These methods take the consideration of signal characteristics. However, power centralizes in low frequency for non-invasive neural signal, therefore signal can be compressed through low pass filter directly. But the high frequency part of invasive BMI includes action potential (spike), which can be decoded effectively. As a consequence, traditional compression methods cannot be directly used in invasive BMI. In this thesis, we explore the characteristics of electroneurographic signal from motor cortex and provide a high fidelity compression framework for it. In this framework, we propose a dual-phase encoding method,  novelly adopt amplitude filter to split the frequency signal from invasive BMI. Low-frequency components use symbol encoding and high-frequency components after quantization uses hybrid encoding method. Hybrid encoding method is combined of Huffman Encoding and Zero-Length Encoding according to the distribution of zeros, then Boundary-Descent Algorithm is put forward to seek the boundary of two methods in hybrid encoding. In the experiment, we compare our proposed method with traditional compression method and typical sequential signal. It shows that our method achieve higher fidelity under same compression ratio, thus can be effectively applied to electroneurographical signal compression and transmission. 

On the other hand, non-invasive BMI provides low-resolution data easily, like Electroencephalograph, which is popular in  neural signal decoding. As a pre task of machine execution, neural signal decoding is the core of BMI. Here we process and detect P300 wave, which is a prework of EEG signal decoding. In this thesis, we apply deep learning method to P300 wavelet detection, establishing ConvP300Net and LSTMP300Net to model EEG signal. ConvP300Net adopts a six layer convolutional neural network, uses local layer to convolute signal in spatial domain, uses convolutional layer to convolute signal in temporal domain, uses fully connect layer to enhance the representational power of our network. Finally, weighted loss is used to measure the classification ability of network. LSTMP300Net adopts a five layer network, in which three hidden layers are Long-short-term memory (LSTM) layer.  Different from forward network, nodes in LSTM are mutually connected in a same layer, so sufficiently combine the characteristics of EEG signal from temporal and spatial domain, thus improve the model and boost detection accuracy. We use Back propagation and Back propagation through time to train ConvP300Net and LSTMP300Net respectively and compare them with related work. Both of them exceed existed work from the perspective of P300 detection performance. 



\englishkeywords{Brain Machine Interface, compression, Convolutional neural network, Long Short Term Memory, Recurrent Neural Network}

\end{englishabstract}
