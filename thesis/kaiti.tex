\documentclass[oneside]{ZJUthesis}

% 该文档中首字符为“%”的均为注释行,不会在论文中出现
% 论文默认为双面模式,需单面模式请将第一行换为如下所示:
% \documentclass[oneside]{ZJUthesis}
% \documentclass[twoside]{ZJUthesis}

% 取消目录中链接的颜色,方便打印
% 如需颜色,请将“false”改为“true”
\hypersetup{colorlinks=false}

% 这里几行代码使得目录中的“第几章” 和后面的章节名称不致发生重叠
\makeatletter
\renewcommand{\numberline}[1]{%
\settowidth\@tempdimb{#1\hspace{0.5em}}%
\ifdim\@tempdima<\@tempdimb%
  \@tempdima=\@tempdimb%
\fi%
\hb@xt@\@tempdima{\@cftbsnum #1\@cftasnum\hfil}\@cftasnumb}
\makeatother

%\usepackage[sectionbib]{chapterbib}
\usepackage{enumerate}
%\usepackage[linesnumbered,boxed]{algorithm2e}


\makeatletter
\def\@chapter[#1]#2{\ifnum \c@secnumdepth >\m@ne
  \if@mainmatter
    \refstepcounter{chapter}%
    \typeout{\@chapapp\space\thechapter.}%
    \addcontentsline{toc}{chapter}%
    {\protect\numberline{第 \chaptername 章}\hspace{1em}#1}
%    {\protect\numberline{\chaptername}\hspace{4em}#1}
  \else
    \addcontentsline{toc}{chapter}{#1}%
  \fi
  \else
  \addcontentsline{toc}{chapter}{#1}%
  \fi
  \chaptermark{#1}%
  \if@twocolumn
  \@topnewpage[\@makechapterhead{#2}]%
  \else
  \@makechapterhead{#2}%
\@afterheading
\fi}
\makeatother

\begin{document}
%%%%%%%%%%%%%%%%%%%%%%%%%%%%%
%% 正文字体设定
%%%%%%%%%%%%%%%%%%%%%%%%%%%%%
\songti

%%%%%%%%%%%%%%%%%%%%%%%%%%%%%
%% 论文封面部分
%%%%%%%%%%%%%%%%%%%%%%%%%%%%%
% 中文封面内容

% 中图分类号
\classification{TM391}

% 单位代码
\serialnumber{10335}

% 密级,如需密级则将其前“%”去掉
\SecretLevel{绝密}
%\SecretLevel{公开}

% 学号
\PersonalID{21221234}

\title{基于侵入式脑电信号压缩和P300波形检测模型的研究}
% 如果标题一行写不下,就写成两行,在下面的命令里写第二行,不需要两行则注释掉
\titletl{论文题目第二行}
%\titletl{test}

%英文题目
\Etitle{English Title The 1st Line}
% 如果一行写不下,同中文题目设定,一行写不下则写两行,不需要就注释掉
\Etitletl{English Title The 2nd Line}
%\Etitletll{}

% 作者
\author{author}
\Eauthor{Author}

%\author{张睿卿}
%\Eauthor{Author}

\degree{硕士}
\Edegree{Master of Engineering}

% 导师
%\supervisor{潘纲教授}
%\Esupervisor{Prof. GangPan}
\supervisor{}
\Esupervisor{}

% 合作导师,如果有的话,去掉注释
% \cpsupervisor{}
% \Ecpsupervisor{}

% 专业名称
\major{计算机科学与技术}
\Emajor{Computer Science and Technology}

% 研究方向
\researchdm{脑机接口}
\Eresearchdm{Brain Machine Interface}

% 所属学院
\institute{计算机科学与技术学院}
\Einstitute{College of Computer Science and Technology}

%论文提交日期
\submitdate{二〇一五年四月三十日}
\Esubmitdate{2015-05-05}

% 答辨日期
\defenddate{2015-05-15}

% 生成封面
\makeCoverPage

% 生成英文封面
\makeECoverPage

% 生成中文题名页
\maketitle



% 生成英文题名页
\makeenglishtitle


%%%%%%%%%%%%%%%%%%%%%%%%%%%%%%
%% 原创声明与版权协议页
%%%%%%%%%%%%%%%%%%%%%%%%%%%%%%

\SignautreDateA{2015}{3}{10}
\SignautreDateB{2015}{3}{10}
\SignautreDateC{2015}{3}{10}
% 生成原创声明与版权协议页
\makeOSandCPRTpage


%%%%%%%%%%%%%%%%%%%%%%%%%%%%%%
%% 论文部分开始
%%%%%%%%%%%%%%%%%%%%%%%%%%%%%%
\ZJUfrontmatter

%%%%%%%%%%%%%%%%%%%%%%%%%%%%%%
%% 勘误页,一般没有
%%%%%%%%%%%%%%%%%%%%%%%%%%%%%%
%\include{./Chapters/corrigenda}


%%%%%%%%%%%%%%%%%%%%%%%%%%%%%%
%% 序言页,一般没有
%%%%%%%%%%%%%%%%%%%%%%%%%%%%%%
%\include{./Chapters/preface}

%%%%%%%%%%%%%%%%%%%%%%%%%%%%%%
%% 摘要
%%%%%%%%%%%%%%%%%%%%%%%%%%%%%%
\begin{abstract}
脑机接口(Brain Machine Interface, BCI)是一个完成大脑神经元和外部机器互联,通信与控制的接口。脑机接口可以代替受损的神经系统,通过大脑信号采集,信号处理,解码为计算机指令,反馈这四步,为残障人士提供自动化服务。本论文主要研究侵入式脑电信号处理,压缩,以及EEG信号中的P300波形检测。

侵入式脑电信号具有高采样率,多通道,高分辨率的性质,因此压缩对于信号存储和传输都是必不可少的环节。 在压缩方面,过去的神经信号压缩工作主要关注于非侵入式脑电信号,这些方法都应用了相关信号的特性。 但这些信号同侵入式信号有很大差异,它们的有效信息都在低频部分,可以直接通过低通滤波器获得有效信号,而侵入式脑机接口获得的信号中高频部分包括神经元锋电位(spike),它能有效地进行解码,因而不能丢弃。所以,EEG等信号的压缩方法不能直接套用在侵入式脑机接口获得的信号中。本文通过侵入式电极研究大脑运动皮层信号性质,利用信道内部神经信号特性建立了一个完整的高保真运动皮层信号压缩框架。
该框架中, 我们提出了一个双阶编码方法, 创新地采用幅值滤波器对侵入式脑电信号的频域数据进行分割, 对于低幅值部分采用符号编码, 对于高幅值部分量化后采用混合编码方法。 混合编码方法由哈弗曼编码和零长编码组合, 根据零分量占比分布分别进行编码, 并针对求两个混合编码的分界点提出了界限下降算法。 在实验中, 我们将双阶编码方法与传统压缩方法和典型时序信号压缩方法进行了比较, 双阶编码方法可以在相同压缩比下达到更高的信号保真度, 从而有效进行脑电神经信号的压缩和传输。

另一方面, 低分辨率的非侵入式脑机接口采样方便, 在信号解码中广泛应用。 只有正确地解码才能实现BCI执行符合用户意愿, 所以神经解码是BCI的核心任务, 我们在本文进行解码的前一步工作, P300波形信号处理与检测。 本文中, 我们采用深度学习的方法, 分别建立了ConvP300Net和LSTMP300Net对EEG信号建模。 ConvP300Net中采用一个6层卷积神经网络, 用局部连接层在空间维度对数据进行非共享卷积, 用卷积层对信号在时间维度进行共享卷积 , 用全连接层增强网络表达能力, 最后采用带权损失函数度量网络分类能力。 LSTMP300Net中采用一个5层网络, 其中3个中间层分别为长短时间记忆(LSTM)层。 不同于属于前向网络的卷积神经网络, LSTM层内各节点相互连接, 可以充分结合EEG信号在时间和空间维度的特性提高检测准确性,完善模型。 我们对ConvP300Net和LSTMP300Net分别用backpropagation和backpropagation through time方法进行训练, 并与相关工作进行比较, 其结果均超过已有工作对P300波形的检测效果。




\keywords{脑机接口,压缩,卷积神经网络,长短期记忆方法,循环神经网络}
\end{abstract}

%{\pagestyle{empty}\mbox{}\newpage\pagestyle{empty}}  % 无页眉页脚的空白页
%%%%%%%%%%%%%%%%%%%%%%%%%%%%%%
%% 英文摘要
%%%%%%%%%%%%%%%%%%%%%%%%%%%%%%
\begin{englishabstract}
Brain Machine Interface (BMI) is an interface that connect neurons of brain and devices to communicate and control. BMI can substitute impaired neural system, provide automatic service to disabled people through four steps: sampling signal from brain, signal processing, signal decoding to computer instruction, and feedback. This thesis focus on invasive electroneurographic signal processing, compression and P300 wavelet detection in Electroencephalograph signal. 

Invasive electroneurographic signal with high sampling rate and multiple channel can represent high-resolution information. However, the huge quantity of data makes compression indispensable. 
In the aspect of signal compression, some related work focus on non-invasive electroneurographic signal. These methods take the consideration of signal characteristics. However, power centralizes in low frequency for non-invasive neural signal, therefore signal can be compressed through low pass filter directly. But the high frequency part of invasive BCI includes action potential (spike), which can be decoded effectively. As a consequence, traditional compression methods cannot be directly used in invasive BCI. In this thesis, we explore the characteristics of electroneurographic signal from motor cortex and provide a high fidelity compression framework for it. 

On the other hand, non-invasive BMI provides low-resolution data easily, like Electroencephalograph, which is popular in  neural signal decoding. As a pre task of machine execution, neural signal decoding is the core of BCI. Here we process and detect P300 wave, which is a prework of EEG signal decoding. In this thesis, we apply deep learning method to P300 wavelet detection, build convolutional neural network and recurrent neural network for EEG signal, and boost detection accuracy from both temporal and spatial dimension.



\englishkeywords{Brain Machine Interface, compression, Convolutional neural network, Long Short Term Memory, Recurrent Neural Network}

\end{englishabstract}

%{\pagestyle{empty}\mbox{}\newpage\pagestyle{empty}}  % 无页眉页脚的空白页

%%%%%%%%%%%%%%%%%%%%%%%%%%%%%%
%% 论文部分开始2
%%%%%%%%%%%%%%%%%%%%%%%%%%%%%%
\ZJUfrontmatterTwo

%%%%%%%%%%%%%%%%%%%%%%%%%%%%%%
%% 目录页
%%%%%%%%%%%%%%%%%%%%%%%%%%%%%% 
\ZJUcontents

%%%%%%%%%%%%%%%%%%%%%%%%%%%%%%
%% 插图列表
%%%%%%%%%%%%%%%%%%%%%%%%%%%%%%
\ZJUListofFigures

%%%%%%%%%%%%%%%%%%%%%%%%%%%%%%
%% 表格列表
%%%%%%%%%%%%%%%%%%%%%%%%%%%%%%
\ZJUListofTables

%%%%%%%%%%%%%%%%%%%%%%%%%%%%%%
%% 缩写、符号清单、术语表
%%%%%%%%%%%%%%%%%%%%%%%%%%%%%%
%\include{./Chapters/symbol}

%%%%%%%%%%%%%%%%%%%%%%%%%%%%%%
%% 正文内容部分开始
%%%%%%%%%%%%%%%%%%%%%%%%%%%%%%
\ZJUmainmatter



\end{document}
