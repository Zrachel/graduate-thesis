\documentclass[oneside]{ZJUthesis}

% 该文档中首字符为“%”的均为注释行,不会在论文中出现
% 论文默认为双面模式,需单面模式请将第一行换为如下所示:
% \documentclass[oneside]{ZJUthesis}
% \documentclass[twoside]{ZJUthesis}

% 取消目录中链接的颜色,方便打印
% 如需颜色,请将“false”改为“true”
\hypersetup{colorlinks=false}

% 这里几行代码使得目录中的“第几章” 和后面的章节名称不致发生重叠
\makeatletter
\renewcommand{\numberline}[1]{%
\settowidth\@tempdimb{#1\hspace{0.5em}}%
\ifdim\@tempdima<\@tempdimb%
  \@tempdima=\@tempdimb%
\fi%
\hb@xt@\@tempdima{\@cftbsnum #1\@cftasnum\hfil}\@cftasnumb}
\makeatother

%\usepackage[sectionbib]{chapterbib}
\usepackage{enumerate}
%\usepackage[linesnumbered,boxed]{algorithm2e}


\makeatletter
\def\@chapter[#1]#2{\ifnum \c@secnumdepth >\m@ne
  \if@mainmatter
    \refstepcounter{chapter}%
    \typeout{\@chapapp\space\thechapter.}%
    \addcontentsline{toc}{chapter}%
    {\protect\numberline{第 \chaptername 章}\hspace{1em}#1}
%    {\protect\numberline{\chaptername}\hspace{4em}#1}
  \else
    \addcontentsline{toc}{chapter}{#1}%
  \fi
  \else
  \addcontentsline{toc}{chapter}{#1}%
  \fi
  \chaptermark{#1}%
  \if@twocolumn
  \@topnewpage[\@makechapterhead{#2}]%
  \else
  \@makechapterhead{#2}%
\@afterheading
\fi}
\makeatother

\begin{document}
%%%%%%%%%%%%%%%%%%%%%%%%%%%%%
%% 正文字体设定
%%%%%%%%%%%%%%%%%%%%%%%%%%%%%
\songti

%%%%%%%%%%%%%%%%%%%%%%%%%%%%%
%% 论文封面部分
%%%%%%%%%%%%%%%%%%%%%%%%%%%%%
% 中文封面内容

% 中图分类号
\classification{TM391}

% 单位代码
\serialnumber{10335}

% 密级,如需密级则将其前“%”去掉
\SecretLevel{绝密}
%\SecretLevel{公开}

% 学号
\PersonalID{21221234}

\title{基于侵入式脑电信号压缩和P300波形检测模型的研究}
% 如果标题一行写不下,就写成两行,在下面的命令里写第二行,不需要两行则注释掉
\titletl{论文题目第二行}
%\titletl{test}

%英文题目
\Etitle{English Title The 1st Line}
% 如果一行写不下,同中文题目设定,一行写不下则写两行,不需要就注释掉
\Etitletl{English Title The 2nd Line}
%\Etitletll{}

% 作者
\author{author}
\Eauthor{Author}

%\author{张睿卿}
%\Eauthor{Author}

\degree{硕士}
\Edegree{Master of Engineering}

% 导师
%\supervisor{潘纲教授}
%\Esupervisor{Prof. GangPan}
\supervisor{}
\Esupervisor{}

% 合作导师,如果有的话,去掉注释
% \cpsupervisor{}
% \Ecpsupervisor{}

% 专业名称
\major{计算机科学与技术}
\Emajor{Computer Science and Technology}

% 研究方向
\researchdm{脑机接口}
\Eresearchdm{Brain Machine Interface}

% 所属学院
\institute{计算机科学与技术学院}
\Einstitute{College of Computer Science and Technology}

%论文提交日期
\submitdate{二〇一五年四月三十日}
\Esubmitdate{2015-05-05}

% 答辨日期
\defenddate{2015-05-15}

% 生成封面
\makeCoverPage

% 生成英文封面
\makeECoverPage

% 生成中文题名页
\maketitle



% 生成英文题名页
\makeenglishtitle


%%%%%%%%%%%%%%%%%%%%%%%%%%%%%%
%% 原创声明与版权协议页
%%%%%%%%%%%%%%%%%%%%%%%%%%%%%%

\SignautreDateA{2015}{3}{10}
\SignautreDateB{2015}{3}{10}
\SignautreDateC{2015}{3}{10}
% 生成原创声明与版权协议页
\makeOSandCPRTpage


%%%%%%%%%%%%%%%%%%%%%%%%%%%%%%
%% 论文部分开始
%%%%%%%%%%%%%%%%%%%%%%%%%%%%%%
\ZJUfrontmatter

%%%%%%%%%%%%%%%%%%%%%%%%%%%%%%
%% 勘误页,一般没有
%%%%%%%%%%%%%%%%%%%%%%%%%%%%%%
%\begin{corrigenda}
这是一个勘误\index{勘误}章节,一般情况下是没有的。
\end{corrigenda}



%%%%%%%%%%%%%%%%%%%%%%%%%%%%%%
%% 序言页,一般没有
%%%%%%%%%%%%%%%%%%%%%%%%%%%%%%
%\begin{preface}
上一版发布于2011年10月26日,发布之后的近两年来,陆陆续续收到一些邮件问关于使用中的一些问题,
我也算基本上做到一一解答。
同进也在着手准备根据提到的问题对这一版模版进行一定的修订,增补一些使用中普遍关心的难点问题。
因为事务冗杂缠身,加上关于参考文献格式调整部分的内容一直没有时间看明白,这个事情就一直拖下来了。
直到前一段断断续续看完了参考文献格式整理部分的帮助资料,搞清楚了它的实现思路原理,
才算又着手修订这一版教程。

在这过去的一年多里,接触到了\XeTeX{},对其强大的直接调用系统字体的能力表示赞叹,
于是将这个模版切换到了\XeTeX{}的环境下,将文件代码换成了对多语言兼容更好的UTF-8代码,
但同时保留对GBK码的兼容,具体不同之处会在后面的章节中提到。
因此新的一版分为UTF-8和GBK两个版本进行发布,两个版本使用上只有很细微的区别,
一般使用过程中可以忽略这个差别。

以下是原来的序言,此处照旧附上。


很早就听说过\LaTeX\index{\LaTeX}了,但却一直没有真正学习过,直到今年,需要处理一些大文档,想起了\LaTeX{}。
重新翻出\LaTeX{}的文档,从CCT开始,至于为什么是CCT,
因为Ctex\index{CTeX}提供的那个CTeX FAQ里对中文的第一个例子,就是以CCT
为例写的。
CCT是中科院的张林波研究员写的,帮助文档都是中文,看起来比较容易,但毕竟是好几年前的版本了,
更新也并不是那么及时,而且CCT\index{CCT}早期版本的字体是点阵字体,边缘很粗糙,
虽然不影响打印,但在这个年代还在用着这样的字体,着实不是那么舒服。
我又开始了第二个例子,CJK的尝试,在尝试CJK\index{CJK}的过程中,
无意中看到了CTeX的ctexart,ctexbook和ctexrep这几个基本模版,这才找到CTeX的门,
筒子们不要笑我绕了这么一大圈才摸进了CTeX的门,虽然从开始就使用的是CTeX的发行版。

这里也要说一下,CTeX提供的部分帮助文档内容也比较老了,一些操作现在新的软件虽然仍然兼容,
但已经不是新版软件推荐的做法了,比如,CTeX FAQ里面对于pdf文件的生成,
依然是先由latex.exe生成dvi文件,再由dvi文件生成ps文件,最后再生成pdf文件。
实际上,现在流行的新版\TeX{}类软件都已经将pdfTeX\index{pdfTeX}作为默认引擎,支持直接生成pdf文件,
而且dvi、ps文件的打开速度比pdf反而要慢许多。我使用的是64位系统,CTeX提供的安装包只支持32位系统,
我单独安装的MikTeX\index{MikTeX} x64\index{x64}版使用CTeX模版生成的dvi文件使用dvips\index{dvips}处理时会找不到字体,
因为这个问题,我找了很久,最后的结论是:dvips可以放弃了,直接使用dvipdfm\index{dvipdfm}更合适。

后来几天在\LaTeX{}的实践中看不少相关细节,开始对其模版产生了兴趣,
在88上\TeX{}版把置顶的ZJUthesis下了下来,就是写这个模版的基础,数学系模版。
下下来后发现这个模板给的例子pdf与当前学校使用的2008年论文模版差别老大了,从封面到目录,
章节格式,都是完全不一样,因此,决定着手做一个与学样提供的Word模版比较接近的模版。

在以2006年数学系模版为基础进行新模版编写的过程中,学了不少方法,
也发现老模版不少过时或者不合适的地方。
第一个学到的就是,从模版一开头就发现这个模版是以ctexbook这个模版为基础制作的,
做到模版完成的时候,
发现88的\TeX{}版置顶模版已经更新,我以为我白做了,
下下来一看,原来这个新的模版不是以ctexbook为基础制作的,而是更基础的\LaTeXe\index{\LaTeX}
对比自己基本完工的模版,才发现ctexbook为我省了很多工作量。只是一些修修改改就做到了很接近学校
word模版的效果。
ctexbook的新版已经直接将hyperref包打了进去,2006年数学系模版对hyperref\index{hyperref}的引用判断部分已经明显示过时,在用新版MikTeX运行的时候直接报错了。
在编写封面的时候,发现2006年的模版用了一个五列的表格,可这部分的内容只需要两列就够了,
直到我某天下载了中科院的模版后才明白,2006年版模版是从中科院模版改编而来,
中科院模版在封面上名字等内容的排列方式需要采用五列表格。这一部分,我也将其重新编写。

随着时代的推进,\LaTeX{}的各种功能包日渐丰富,很多过去只能从\LaTeXe{}代码写的功能,
如今可以通过相应的功能包直接实现,在这个模版中,我使用了几个新的功能包,
其中最新的当属刚刚发布的hyperref更新包,增加了hidelinks命令,可以直接将链接的边框去掉,
不用采用将边框颜色设为白色的方式了。

就像\LaTeX{}的版本总是在接近$\pi$的值一样,这份模版并不是完美的,比如对数学系的定理体系支持不足,
留在以后版本再发布或者请有兴趣的爱好者共同修改。编写这一版本的基本目的是没有任何\LaTeX{}基础的同学可以比较轻松地利用它给自己的毕业论文排一个满意的版面,整个模版没有留太多选项,
可供修改的选项只有两个:单面双面的选择和链接的颜色的有无。在模版中,我对绝大多数的语句,
都做了中文注释,解释其作用,方便有兴趣的同学研究,我也是一个初学者,作出的这份模版,
我想,应该是比较适合初学者胃口的。

\end{preface}


%%%%%%%%%%%%%%%%%%%%%%%%%%%%%%
%% 摘要
%%%%%%%%%%%%%%%%%%%%%%%%%%%%%%
\begin{abstract}

多年来人们在生物学医学领域的研究发现, 生物脑电活动可以通过一个非肌肉通路向外设传递信息, 这个外设被称作脑机接口。 脑机接口可以代替受损的神经系统,通过大脑信号采集,信号处理,解码为计算机指令,反馈这四步,为残障人士提供自动化服务。本论文主要研究侵入式脑电信号处理,压缩,以及EEG信号中的P300波形检测。

侵入式脑机接口在神经手术中直接植入大脑灰质, 其采集的信号具有高分辨率, 高采样率的性质,因此压缩对于信号存储和传输都是必不可少的环节。 在压缩方面,过去的神经信号压缩工作主要关注于非侵入式脑电信号,这些方法都应用了相关信号的特性。 但这些信号同侵入式信号有很大差异,所以其压缩方法不能直接套用在侵入式脑电信号中。 另一方面, 低分辨率的非侵入式脑机接口采样方便, 如脑电图(EEG)在大脑头皮非侵入式地记录神经元内离子电流产生的电位波动\cite{nie2005ele}, 所采集的信号在神经解码中广泛应用。 P300波形是大脑在决策时产生的事件相关电位中的正向偏移, 表示年轻成年人对简单的感官可分辨目标响应峰值延时约300ms。 

本文的主要工作如下:
\begin{enumerate}
\item 通过侵入式电极研究大脑运动皮层信号性质, 利用信道内部神经信号特性建立了一个完整的高保真运动皮层信号压缩框架。 该框架称为双阶编码方法, 创新地采用幅值滤波器对侵入式脑电信号的频域数据进行分割, 提出符号编码和混合编码方法分别对分割后数据进行压缩。 混合编码方法由哈弗曼编码和零长编码组合, 我们根据零分量占比分布提出了界限下降算法来求取两种编码方法的分界点。 在实验中, 我们将双阶编码方法与传统压缩方法和典型时序信号压缩方法进行了比较, 双阶编码方法可以在相同压缩率下达到更高的信号保真度, 在18\%的压缩比下达到36db的信噪比, 并保留了92\%的动作电位信号。

\item 采用深度学习的方法进行P300波形信号处理与检测。 文中, 我们分别建立了ConvP300Net和LSTMP300Net对EEG信号建模。 ConvP300Net中提出用一个6层卷积神经网络, 用局部连接层在空间维度对数据进行非共享卷积, 用卷积层对信号在时间维度进行共享卷积 , 用全连接层增强网络表达能力, 最后采用带权损失函数度量网络分类能力。 LSTMP300Net中采用一个5层网络, 其中3个中间层分别为长短时间记忆(LSTM)层。 不同于属于前向网络的卷积神经网络, LSTM层内各节点相互连接, 可以充分结合EEG信号在时间和空间维度的特性提高检测准确性,完善模型。 我们对ConvP300Net和LSTMP300Net分别用backpropagation和backpropagation through time方法进行训练, 并与相关工作进行比较, 结果ConvP300Net和LSTMP300Net在BCI竞赛数据集上分别达到80.09\%和84.47\%的识别率, 均超过已有工作对P300波形的检测效果。

\end{enumerate}





\keywords{脑机接口,压缩,卷积神经网络,长短期记忆方法,循环神经网络}
\end{abstract}

%{\pagestyle{empty}\mbox{}\newpage\pagestyle{empty}}  % 无页眉页脚的空白页
%%%%%%%%%%%%%%%%%%%%%%%%%%%%%%
%% 英文摘要
%%%%%%%%%%%%%%%%%%%%%%%%%%%%%%
\begin{englishabstract}
Brain Machine Interface (BMI) is an interface that connect neurons of brain and devices to communicate and control. BMI can substitute impaired neural system, provide automatic service to disabled people through four steps: sampling signal from brain, signal processing, signal decoding to computer instruction, and feedback. This thesis focus on invasive electroneurographic signal processing, compression and P300 wavelet detection in Electroencephalograph signal. 

Invasive electroneurographic signal with high sampling rate and multiple channel can represent high-resolution information. However, the huge quantity of data makes compression indispensable. 
In the aspect of signal compression, some related work focus on non-invasive electroneurographic signal. These methods take the consideration of signal characteristics. However, power centralizes in low frequency for non-invasive neural signal, therefore signal can be compressed through low pass filter directly. But the high frequency part of invasive BCI includes action potential (spike), which can be decoded effectively. As a consequence, traditional compression methods cannot be directly used in invasive BCI. In this thesis, we explore the characteristics of electroneurographic signal from motor cortex and provide a high fidelity compression framework for it. 

On the other hand, non-invasive BMI provides low-resolution data easily, like Electroencephalograph, which is popular in  neural signal decoding. As a pre task of machine execution, neural signal decoding is the core of BCI. Here we process and detect P300 wave, which is a prework of EEG signal decoding. In this thesis, we apply deep learning method to P300 wavelet detection, build convolutional neural network and recurrent neural network for EEG signal, and boost detection accuracy from both temporal and spatial dimension.



\englishkeywords{Brain Machine Interface, compression, Convolutional neural network, Long Short Term Memory, Recurrent Neural Network}

\end{englishabstract}

%{\pagestyle{empty}\mbox{}\newpage\pagestyle{empty}}  % 无页眉页脚的空白页

%%%%%%%%%%%%%%%%%%%%%%%%%%%%%%
%% 论文部分开始2
%%%%%%%%%%%%%%%%%%%%%%%%%%%%%%
\ZJUfrontmatterTwo

%%%%%%%%%%%%%%%%%%%%%%%%%%%%%%
%% 目录页
%%%%%%%%%%%%%%%%%%%%%%%%%%%%%% 
\ZJUcontents

%%%%%%%%%%%%%%%%%%%%%%%%%%%%%%
%% 插图列表
%%%%%%%%%%%%%%%%%%%%%%%%%%%%%%
\ZJUListofFigures

%%%%%%%%%%%%%%%%%%%%%%%%%%%%%%
%% 表格列表
%%%%%%%%%%%%%%%%%%%%%%%%%%%%%%
\ZJUListofTables

%%%%%%%%%%%%%%%%%%%%%%%%%%%%%%
%% 缩写、符号清单、术语表
%%%%%%%%%%%%%%%%%%%%%%%%%%%%%%
%\begin{ListofSymbol}
缩写、符号清单、术语表
\end{ListofSymbol}


%%%%%%%%%%%%%%%%%%%%%%%%%%%%%%
%% 正文内容部分开始
%%%%%%%%%%%%%%%%%%%%%%%%%%%%%%
\ZJUmainmatter



\end{document}
